\section{A Rectangle Packing Formulation for BEB Charging}  \label{sec:problemformulation}

Applying the PAP to BEB charging requires four fundamental changes. The first is that the time that a BEB spends
charging is allowed to vary. Thus, $s_i$ becomes a variable of optimization. Second, in the PAP each charging visit is
assumed to be a different vehicle. For the BEB charging problem, each bus may make multiple visits to the station
throughout the day and the resulting charge for a bus at a given time is dependent upon each of the prior visits made.
The third fundamental change is related to the first two. The charge of each bus must be tracked in
the optimization to ensure that charging across multiple visits is sufficient to allow each bus to execute its route throughout the day.
The final change in the PAP, the charger is one continuous bar with vehicle width effectively restricting the number of vehicles
charging simultaneously. For the BEB, it is assumed that a discrete number of chargers exist. Moreover, it is assumed
that these chargers have different charge rates.

The discussion of the four changes are separated into two sections. \autoref{sec:queuing} discusses the changes in the
spatial-temporal constraint formulation to form a queuing constraint. \autoref{sec:batt_dynamics} then discusses the
addition of the bus charge management. This section ends with a brief discussion of a modified objective function and
the statement of the full problem in \autoref{sec:BEB_MILP}. The notation is explained throughout and summarized in
\autoref{tab:variables}.

\subsection{Queuing Constraints} \label{sec:queuing}

\noindent
The queuing constraints ensure that the busses entering the queues for charging are assigned in a feasible manner as
they come into the station. There are three sets to differentiate between different entities. $\mathbb{B} = \{1, ...,
n_B\}$ is the set of bus indices with index $b$ used to denote an individual bus, $\mathbb{Q} = \{1, ..., n_Q\}$ is the set of
queues with index $q$ used to denote an individual queue, and $\mathbb{V} = \{1, ..., n_V\}$ is a set of visits to the
station with $i,j$ used to refer to individual visits. The mapping $\Gamma: \mathbb{V} \rightarrow \mathbb{B}$ is used to map a visit
index to a bus index with the shorthand $\Gamma_i$ used to refer to the bus index for visit $i$.

Most variables are now defined in terms of a visit. Two separate visits could correspond to different buses or visits by
the same bus. The PAP spatial variable, $s_i$, is removed and $v_i$ is made to be an integer corresponding to which
queue visit $i$ will be using. Thus, when $\psi_{ij} = 1$, vehicle $i$ is queued to a charger that has a larger index than
the charger that vehicle $j$ is queued, i.e., $v_i-v_j \geq 1$. The variable $S$ is likewise replaced with $n_Q$. Note that
$n_Q = n_B + n_C$, where $n_B$ is the number of busses and $n_C$ is the number of chargers. The rationale for having
more queues than chargers is to allow buses to sit idle instead of requiring the bus to charge at each visit. The
modified queuing constraints can be written as shown in \autoref{eq:packconstrs}.

\begin{subequations}
\label{eq:packconstrs}
\begin{align}
    v_i - v_j - (\psi_{ij} - 1)n_Q \geq 1       \label{subeq:space}        \\
    d_i \leq \tau_i                             \label{subeq:valid_depart} \\
    s_i \geq 0                               \label{subeq:pos_charge} \\
    v_i \in \mathbb{Q}                       \label{subeq:vspace}
\end{align}
\end{subequations}

The constraint in \autoref{subeq:space} is nearly identical to \autoref{subeq:bapspace}, but rather than viewing the
charger as a continuous strip of length $S$, it is discretized into $n_Q$ queues a width of unit length one. A BEB is
also assigned a unit length of one which is reflected in \autoref{subeq:space} by $\cdot \geq 1$. \autoref{subeq:valid_depart}
ensures that the time the BEB is detatched from the charger, $d_i$, is before its departure time, $\tau_i$.
\autoref{subeq:vspace} defines the integer set of indices for queues for $v_i$.

\subsection{Battery Charge Dynamic Constraints}
\label{sec:batt_dynamics}
Battery dynamic constraints are now introduced to relate busses to visits and guarantee that buses have sufficient time
to charge. Two constraints are enforced on the bus charge: busses must always have sufficient charge to execute their
respective routes and each bus must end the day with a specific charge threshold, preparatory for the next day.

The charge at the beginning of visit $i$ is denoted as $\eta_i$. As a charge on the bus is dependent upon the visits that
bus makes to the station, the mapping $\Upsilon: \mathbb{V} \rightarrow \mathbb{V} \bigcup \{\varnothing\}$ is used to determine the next visit
that corresponds to the same bus, with $\Upsilon_i$ being shorthand notation. Thus, $\Gamma_i$ and $\Gamma_{\Upsilon_i}$ would both map to the
same bus index as long as $\Upsilon_i$ is not the null element, $\varnothing$. That is, $\Gamma_{\Upsilon_i}$ where $\Upsilon_i = 0$ indicates
that there are no future visits for bus $i$.

To drive time spent on the charger, $s_i$, as well as define initial, final, and intermediate bus charges for each visit
$i$, the sets for initial and final visits must be defined. Let the mapping of the first visit by each bus be denoted as
$\Gamma^0_i : \mathbb{B} \rightarrow \mathbb{V}$. The indexed value of $\Gamma^0_i$ represents the index for the first visit of bus $b$ or
the null element, $\varnothing$. Similarly, let $\Gamma^f_i : \mathbb{B} \rightarrow \mathbb{B}$ contain the indexes for the final
visit of each bus $b$ or the null element. The initial and final bus charge percentages, $\alpha$ and $\beta$, can then be
represented by the constraint equations $\eta_{\Gamma^0_i} = \alpha \kappa_{\Gamma^0_i}$ and \(\eta_{\Gamma^f_i} = \beta \kappa_{\Gamma^f_i}\), respectively. The
intermediate charges must be determined at solve time.

It is assumed that the charge received is proportional to the time spent charging. The charge rate for charger $q$ is
denoted as $r_q$. Note that a value of $r_q = 0$ corresponds to a queue where no charging occurs. A bus in such a queue
is simply waiting for the departure time. The queue indices are ordered such that the final $n_B$ queues have $r_q = 0$
to allow an arbitrary number of buses to sit idle at any given moment in time. The amount of discharge between visits
$i$ and $\Upsilon_i$, the next visit of the same bus, is denoted as $\Delta_i$. If visit $i$ occurred at charger $q$, the charge of
the bus coming into visit $\Upsilon_i$ would be $\eta_{\Upsilon_i} = \eta_i + s_i r_q - \Delta_i$.

The binary decision variable $w_{iq}$ is introduced to determine whether visit $i$ uses charger $q$. This allows the
charge of the bus coming into visit $\Upsilon_i$ to be written in summation form as

\begin{subequations}
    \label{subeq:pre_next_charge}
\begin{align}
    \eta_{\Upsilon_i} = \eta_i + \sum_{q=1}^{n_Q} s_i w_{iq} r_q - \Delta_i  \\
    \sum_{q=1}^{n_Q} w_{iq} = 1 \\
    w_{iq} \in \{0,1\}
\end{align}
\end{subequations}

The choice of queue for visit $i$, becomes a slack variable and is defined in terms of $w_{iq}$ as

\begin{equation}
    v_i = \sum_{q=1}^{n_Q} qw_{iq}
\end{equation}

Maximum and minimum values for the charges are included to ensure that the battery is not overcharged and to guarantee
sufficient charge for subsequent visits. The upper and lower battery charge bounds for bus $b$ are $\kappa_b$ and $\nu_b \kappa_b$,
respectively0 $\kappa_b$ is the battery capacity and $\nu_b$ is a percent value. As $\eta_i$ corresponds to the charge at the
beginning of the visit, the upper bound constraint must also include the charge received during the visit as follows.

\begin{subequations}
    \label{subeq:pre_min_max}
\begin{align}
    \eta_i + \sum_{q=1}^{n_Q} s_i w_{iq} r_q \leq \kappa_{\Gamma_i}                 \\
    \eta_i \geq \nu_{\Gamma_i} \kappa_{\Gamma_i}
\end{align}
\end{subequations}

Note that the term $s_i w_{iq}$ is a bilinear term. A standard way of linearizing a bilinear term that contains an
integer variable is by introducing a slack variable with an either/or constraint
\cite{chen-2010-applied,rodriguez-2013-compar-asses}. Allowing the slack variable $g_{iq}$ to be equal to $s_i w_{iq}$,
$g_{iq}$ can be defined as

\begin{equation}
    \label{eq:giq_cases}
    g_{iq} =
    \begin{cases}
        s_i & w_{iq} = 1 \\
        0 & w_{iq} = 0
    \end{cases}.
\end{equation}

\autoref{eq:giq_cases} can be expressed as a mixed integer constraint using big-M notation with the following four
constraints.

\begin{subequations}
    \label{eq:slack_gain}
\begin{align}
    s_i - (1 - w_{iq})M \leq g_{iq}  \label{subeq:repgpgret} \\
    s_i \geq g_{iq}                 \label{subeq:repgples} \\
    Mw_{iq} \geq g_{iq}              \label{subeq:repgwgret} \\
    0 \leq g_{iq}                   \label{subeq:repgwles}
\end{align}
\end{subequations}

\noindent
where $M$ is a large value. If $w_{iq} = 1$ then \autoref{subeq:repgpgret} and \autoref{subeq:repgples} become $s_i \leq
g_{iq}$ and $s_i \geq g_{iq}$, forcing $s_i = g_{iq}$ with \autoref{subeq:repgwgret} being inactive. If $w_{iq} = 0$,
\autoref{subeq:repgpgret} is inactive and \autoref{subeq:repgwgret} and \autoref{subeq:repgwles} force $g_{iq} = 0$.

\subsection{The BEB Charging Problem} \label{sec:BEB_MILP}
The goal of the MILP is to utilize chargers as little as possible to reduce energy costs with the fast charging being
penalized more to reduce battery damage. Thus, an assignment cost $m_q$ and usage cost $\epsilon_q$ are associated with each
charger, $q$. These weights can be adjusted based on charger type or time of day that the visit occurs. The assignment
term takes the form $w_{iq}m_q$, and the usage term takes the form $g_{iq} \epsilon_q$. The resulting BEB charging problem is
defined in \autoref{eq:objective}.

\begin{equation}
\label{eq:objective}
	\min \sum_{i=1}^N \sum_{q=1}^{n_Q} \Big( w_{iq} m_q + g_{iq} \epsilon_q \Big) \\
\end{equation}

Subject to the constraints

\begin{multicols}{2}
\begin{subequations}
                                                     \label{eq:dynconstrs}
\begin{equation}
    u_i - u_j - s_j - (\sigma_{ij} - 1)T \geq 0              \label{subeq:m_time}         \\
\end{equation}
\begin{equation}
    v_i - v_j - (\psi_{ij} - 1)n_Q \geq 1                  \label{subeq:m_space}        \\
\end{equation}
\begin{equation}
    \sigma_{ij} + \sigma_{ji} + \psi_{ij} + \psi_{ji} \geq 1            \label{subeq:m_valid_pos}    \\
\end{equation}
\begin{equation}
    \sigma_{ij} + \sigma_{ji} \leq 1                              \label{subeq:m_sigma}        \\
\end{equation}
\begin{equation}
    \psi_{ij} + \psi_{ji} \leq 1                              \label{subeq:m_delta}        \\
\end{equation}
\begin{equation}
    s_i + u_i = d_i                                  \label{subeq:m_detach}       \\
\end{equation}
\begin{equation}
    \eta_{\Gamma^0_i} = \alpha \kappa_{\Gamma^0_i}                           \label{subeq:init_charge}    \\
\end{equation}
\begin{equation}
    a_i \leq u_i \leq (T - s_i)                            \label{subeq:m_valid_starts} \\
\end{equation}
\begin{equation}
    d_i \leq \tau_i                                        \label{subeq:m_valid_depart} \\
\end{equation}
\begin{equation}
    \eta_i + \sum_{q=1}^{n_Q} g_{iq} r_q - \Delta_i = \eta_{\gamma_i}   \label{subeq:next_charge}    \\
\end{equation}
\begin{equation}
    \eta_i + \sum_{q=1}^{n_Q} g_{iq} r_q - \Delta_i \geq \nu \kappa_{\Gamma_i} \label{subeq:min_charge}     \\
\end{equation}
\begin{equation}
    \eta_i + \sum_{q=1}^{n_Q} g_{iq} r_q \leq \kappa_{\Gamma_i}         \label{subeq:max_charge}     \\
\end{equation}
\begin{equation}
    \eta_{\Gamma^f_i} \geq \beta \kappa_{\Gamma^f_i}                          \label{subeq:final_charge}   \\
\end{equation}
\begin{equation}
    s_i - (1 - w_{iq})M \leq g_{iq}                     \label{subeq:gpgret}         \\
\end{equation}
\begin{equation}
    s_i \geq g_{iq}                                     \label{subeq:gples}          \\
\end{equation}
\begin{equation}
    Mw_{iq} \geq g_{iq}                                 \label{subeq:gwgret}         \\
\end{equation}
\begin{equation}
    0 \leq g_{iq}                                       \label{subeq:gwles}          \\
\end{equation}
\begin{equation}
    v_i = \sum_{q=1}^{n_Q} qw_{iq}                      \label{subeq:wmax}           \\
\end{equation}
\begin{equation}
    \sum_{q=1}^{n_Q} w_{iq} = 1                         \label{subeq:wone}           \\
\end{equation}
\begin{equation}
   w_{iq}, \sigma_{ij}, \psi_{ij} \in \{0,1\}\;            \label{subeq:binaryspace}        \\
\end{equation}
\begin{equation}
    v_i, q_i \in  \mathbb{Q}                                         \label{subeq:Qspace}        \\
\end{equation}
\begin{equation}
    i \in \mathbb{V}                                   \label{subeq:Ispace}         \\
\end{equation}
\end{subequations}
\end{multicols}

\autoref{subeq:m_time}-\autoref{subeq:m_valid_depart} are reiterations of the queuing constraints in
\autoref{eq:packconstrs}. \autoref{subeq:init_charge}-\autoref{subeq:final_charge} provide the battery charge
constraints. \autoref{subeq:gpgret} through \autoref{subeq:gwles} define the charge gain of every visit/queue
pairing. The last constraints \autoref{subeq:binaryspace}-\autoref{subeq:Ispace} define the sets of valid values for each
variable.

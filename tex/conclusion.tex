\section{Conclusion}
\label{sec:conclusion}

This work developed a MILP scheduling framework that optimally assigns slow and fast chargers to a BEB bus fleet
assuming a constant schedule. The BAP was introduced with an example formulation and was then compared to the PAP. The
PAP constructed on the BAP to allow the time spent on the charger, $s_i$, to be a decision variable. Because the
original PAP required service time, $s_i$, to be given, linear battery dynamics were introduced to drive charging times.
Additional constraints were also introduced to provide limits for the battery dynamics.

An example for the MILP PAP formulation was then presented and compared to a heuristic based schedule, referred to as
Quin-Modified. The MILP PAP optimization was run for \timeran seconds to a non-optimal solution with a gap of
\gappercent. The Quin-Modified and MILP schedule utilized similar amount of fast chargers; however, the MILP schedule
more heavily used the slow chargers to its advantage when the time was available to do so. More importantly, the MILP
PAP schedule utilized approximately $1\cdot10^5$ Kwh more than the Quin-Modified, but the charges remained above the
constrained minimum charge of \mincharge, and charged all the buses to \fpeval{\bcharge *100}\% at the end of the
working day. The Quin-Modified schedule, on the other hand, failed to keep all the bus charges above 0\% throughout the
time horizon.

Further fields of interest are to utilize the formulation (\autoref{eq:objective} and \autoref{eq:dynconstrs}) with
nonlinear battery dynamics, calculation and utilization of the demand and consumption cost in the objective function,
and utilizing this formulation in a metaheuristic solver. Furthermore, ``fuzzifying'' the initial and final charge times
is of interest to allow flexibility in the arrival and departure times.

%%%%%%%%%%%%%%%%%%%%%%%%%%%%%%%%%%%%%%%%%%%%%%%%%%%%%%%%%%%%%%%%%%%%%%%%%%%%%%%%%%
% LINKS:
%
% https://github.com/maxbw117/DevelopmentPerSecond/blob/master/Tikz-pgfplots-and-latex/Tutorial%202-%20Figures%20and%20Large%20File%20Organization/Figures%20Chapter%201/01%20Ocean%20and%20Model%20Scale.tex
% https://www.overleaf.com/learn/latex/Questions/I_have_a_lot_of_tikz%2C_matlab2tikz_or_pgfplots_figures%2C_so_I%27m_getting_a_compilation_timeout._Can_I_externalise_my_figures%3F
%%%%%%%%%%%%%%%%%%%%%%%%%%%%%%%%%%%%%%%%%%%%%%%%%%%%%%%%%%%%%%%%%%%%%%%%%%%%%%%%%%

\section*{Figure Captions}

%%--------------------------------------------------------------------------------
% BAP and PAP comparison
\begin{subfigures}
    %%~~~~~~~~~~~~~~~~~~~~~~~~~~~~~~~~~~~~~~~~~~~~~~~~~~~~~~~~~~~~~~~~~~~~~~~~~~~~
    % BAP
    \begin{figure}[htpb]
    \centering
        \includestandalone{img/bap}
        \caption{Example of berth allocation. Vessels are docked in berth locations (horizontal) and are queued over
          time (vertical). The vertical arrow represents the movement direction of queued vessels and the horizontal
          arrow represents the direction of departure.}
        \label{subfig:bapexample}
    \end{figure}
    \hfill

    %%~~~~~~~~~~~~~~~~~~~~~~~~~~~~~~~~~~~~~~~~~~~~~~~~~~~~~~~~~~~~~~~~~~~~~~~~~~~~
    % PAP
    \begin{figure}[htpb]
    \centering
        \includestandalone{img/pap}
        \caption{Example of position allocation. Vehicles are placed in queues to be charged and move in the direction
          indicated by the arrow.}
        \label{subfig:papexample}
    \end{figure}
\end{subfigures}


%%--------------------------------------------------------------------------------
% Variable table
\begin{table}[!htpb]
  \caption{Notation used throughout the paper}
  \label{tab:variables}
  \centering
  \begin{tabularx}{\textwidth}{l l}
    \toprule
    \textbf{Variable} & \textbf{Description}                                                                               \\
    \toprule
    \multicolumn{2}{l}{Input values}                                                                                       \\
    $n_B$        & Number of buses                                                                                         \\
    $M$          & An arbitrary very large upper bound value                                                               \\
    $n_V$        & Number of total visits                                                                                  \\
    $n_Q$        & Number of queues                                                                                        \\
    $n_C$ 	 & Number of chargers                                                                                      \\
    $\mathbb{V}$ & Set of visit indices, $\mathbb{V} = \{1, ..., n_V\}$                                                    \\
    $\mathbb{B}$ & Set of bus indices, $\mathbb{B} = \{1, ..., n_B\}$                                                      \\
    $\mathbb{Q}$          & Set of queue indices, $\mathbb{Q} = \{1, ..., n_Q\}$                                                             \\
    $i,j$        & Indices used to refer to visits                                                                         \\
    $b$ 	 & Index used to refer to a bus                                                                            \\
    $q$ 	 & Index used to refer to a queue                                                                          \\
    \hline
    \multicolumn{2}{l}{Problem definition parameters}                                                                      \\
    $\Gamma$   & $\Gamma: \mathbb{V} \rightarrow \mathbb{B}$ with $\Gamma_i$ used to denote the bus for visit $i$                                   \\
    $\alpha_i$ & Initial charge percentage time for visit $i$                                                                   \\
    $\beta_i$ & Final charge percentage for bus $i$ at the end of the time horizon                                             \\
    $\epsilon_q$ & Cost of using charger $q$ per unit time                                                                        \\
    $\Upsilon$   & $\Upsilon: \mathbb{V} \rightarrow \mathbb{V}$ mapping a visit to the next visit by the same bus with $\Upsilon_i$ being the shorthand. \\
    $\kappa_b$ & Battery capacity for bus $b$                                                                                   \\
    $\Delta_i$ & Discharge of visit over route $i$                                                                              \\
    $\nu_b$ & Minimum charge allowed for bus $b$                                                                             \\
    $\tau_i$ & Time visit $i$ must depart the station                                                                         \\
    $\zeta_b$ & Discharge rate for bus $b$                                                                                     \\
    $a_i$ & Arrival time of visit  $i$                                                                                     \\
    $i_0$ & Indices associated with the initial arrival for every bus in $A$                                               \\
    $i_f$ & Indices associated with the final arrival for every bus in $A$                                                 \\
    $m_q$ & Cost of a visit being assigned to charger $q$                                                                  \\
    $r_q$ & Charge rate of charger $q$ per unit time                                                                       \\
    \hline
    \multicolumn{2}{l}{Decision Variables}                                                                                 \\
    $\psi_{ij}$ & Binary variable determining spatial ordering of vehicles $i$ and $j$                                       \\
    $\eta_i$    & Initial charge for visit $i$                                                                                \\
    $\sigma{ij}$ & Binary variable determining temporal ordering of vehicles $i$ and $j$                                       \\
    $d_i$    & Ending charge time for visit $i$                                                                            \\
    $g_{iq}$ & The charge gain for visit $i$ from charger $q$                                                              \\
    $s_i$    & Amount of time spent on charger for visit $i$                                                               \\
    $u_i$    & Starting charge time of visit $i$                                                                           \\
    $v_i$    & Assigned queue for visit $i$                                                                                \\
    $w_{iq}$ & Binary assignment variable for visit $i$ to queue $q$                                                       \\
    \bottomrule
  \end{tabularx}
\end{table}


%%--------------------------------------------------------------------------------
% Berth spatio-temporal graph
\begin{figure}[htpb]
\centering
    \includegraphics{img/spatiotemporal-packing}
    \caption{Example of rectangle packing problem.}
    \label{fig:packexample}
\end{figure}

%%--------------------------------------------------------------------------------
% Spatial temporal graph representation
\begin{figure}[ht]
\centering
    \includegraphics{img/baprep}
    \caption{The representation of the berth-time space}
    \label{fig:bap}
\end{figure}

%%--------------------------------------------------------------------------------
% Spatio-temporal overlap visualization
\begin{figure}[htpb]
\centering
    \includegraphics{img/overlap}
    \caption{Examples of different methods of overlapping. Space overlap: $v_{k_1} < v_{i} + s_i \therefore \delta_{k_{1}i} = 0$.
             Time overlap $u_{k_1} < u_{j} + p_j \therefore \sigma_{k_{2}j} = 0$. Both space and time overlap $\sigma_{k_{3}i} = 0$ and
             $\delta_{k_{3}j} = 0$.}
    \label{fig:multipleassign}
\end{figure}

%%--------------------------------------------------------------------------------
% Charge schedule
\begin{subfigures}
    %%~~~~~~~~~~~~~~~~~~~~~~~~~~~~~~~~~~~~~~~~~~~~~~~~~~~~~~~~~~~~~~~~~~~~~~~~~~~~
    % Quinn
    \begin{figure}[htpb]
    \centering
        \includegraphics{img/schedule-quinn}
        \caption{Charging schedule generated by Quin Modified algorithm.}
        \label{subfig:quin-schedule}
    \end{figure}

    \hfill

    %%~~~~~~~~~~~~~~~~~~~~~~~~~~~~~~~~~~~~~~~~~~~~~~~~~~~~~~~~~~~~~~~~~~~~~~~~~~~~
    % MILP
    \begin{figure}[htpb]
    \centering
        \includegraphics{img/schedule-milp}
        \caption{Charging schedule generated by MILP PAP algorithm.}
        \label{subfig:milp-schedule}
    \end{figure}
\end{subfigures}

%%--------------------------------------------------------------------------------
% Bus charges
\begin{subfigures}
    %%~~~~~~~~~~~~~~~~~~~~~~~~~~~~~~~~~~~~~~~~~~~~~~~~~~~~~~~~~~~~~~~~~~~~~~~~~~~~
    % Quinn
    \begin{figure}[htpb]
    \centering
        \includegraphics{img/charge-quinn}
        \caption{Bus charges for the Quin Modified charging schedule. The charging scheme of the Quin charger is more predictable during the working day.}
        \label{subfig:quin-charge}
    \end{figure}

    \hfill

    %%~~~~~~~~~~~~~~~~~~~~~~~~~~~~~~~~~~~~~~~~~~~~~~~~~~~~~~~~~~~~~~~~~~~~~~~~~~~~
    % MILP
    \begin{figure}[htpb]
    \centering
        \includegraphics{img/charge-milp}
        \caption{The bus charges for the MILP PAP charging schedule. The MILP model allows for guarantees of minimum/maximum changes during the working day as well as charges at the end of the day.}
        \label{subfig:milp-charge}
    \end{figure}
\end{subfigures}

%%--------------------------------------------------------------------------------
% Power consumption
\begin{figure}[htpb]
\centering
    \includegraphics{img/power}
    \caption{Total accumulated energy consumed by the Quin-Modified and MILP schedule throughout the time horizon.}
    \label{fig:power-usage}
\end{figure}

%%--------------------------------------------------------------------------------
% Energy use
\begin{figure}[htpb]
\centering
    \includegraphics{img/energy}
    \caption{Amount of power consumed by Quin-Modified and MILP schedule over the time horizon.}
    \label{fig:energy-usage}
\end{figure}

%%--------------------------------------------------------------------------------
% Charger usage count
\begin{subfigures}
    %%~~~~~~~~~~~~~~~~~~~~~~~~~~~~~~~~~~~~~~~~~~~~~~~~~~~~~~~~~~~~~~~~~~~~~~~~~~~~
    % Fast
    \begin{figure}[htpb]
    \centering
        \includegraphics{img/charger-count-fast}
        \caption{Number of fast chargers for Quin and MILP PAP.}
        \label{subfig:fast-charger-usage}
    \end{figure}

    \hfill

    %%~~~~~~~~~~~~~~~~~~~~~~~~~~~~~~~~~~~~~~~~~~~~~~~~~~~~~~~~~~~~~~~~~~~~~~~~~~~~
    % Slow
    \begin{figure}[!ht]
    \centering
        \includegraphics{img/charger-count-slow}
        \caption{Number of slow chargers for Quin and MILP PAP.}
        \label{subfig:slow-charger-usage}
    \end{figure}
\end{subfigures}

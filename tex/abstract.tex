\begin{abstract}
  Robust charging schedules for an increasing interest of battery electric bus (BEB) fleets is a critical component to
  successful adoption. In this paper, a BEB charging scheduling framework that considers spatiotemporal schedule
  constraints, route schedules, fast and slow charging, and battery dynamics is modeled as a mixed integer linear
  program (MILP). The MILP is modeled after the berth allocation problem (BAP) in a modified form known as the position
  allocation problem (PAP). Linear battery dynamics are included to model the charging and discharging of buses while at
  the station and during their routes, respectively. The optimization coordinates BEB charging to ensure each BEB has
  sufficient charge while using slow chargers where possible for sake of battery health. The model also minimizes the
  total number of chargers utilized and prioritizes slow chargers. The model validity is demonstrated with a set of
  routes sampled from \TODO{Utah Transit Authority (UTA)} for \A buses and \N visits to the charging station. The model
  is also compared to a threshold based algorithm, referred to as the Quin-Modified method. The results presented show
  that the slow chargers are more readily selected and the charging and spatiotemporal constraints are met while
  considering the battery dynamics, minimizing the charger count, and consumption cost.

  \tiny \keyFont{ \section{Keywords:} Berth Allocation Problem (BAP), Position Allocation Problem (PAP), Mixed Integer
    Linear Program (MILP), Battery Electric Bus (BEB), Scheduling} %All article types: you may provide up to 8 keywords;
  at least 5 are mandatory.
\end{abstract}

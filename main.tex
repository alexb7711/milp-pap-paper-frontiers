% Created 2023-08-12 Sat 18:25
% Intended LaTeX compiler: pdflatex
\documentclass[utf8]{FrontiersinHarvard}
\usepackage[utf8]{inputenc}
\usepackage[T1]{fontenc}
\usepackage{graphicx}
\usepackage{longtable}
\usepackage{wrapfig}
\usepackage{rotating}
\usepackage[normalem]{ulem}
\usepackage{amsmath}
\usepackage{amssymb}
\usepackage{capt-of}
\usepackage{hyperref}
%%==============================================================================
% Package configuration
\usepackage{amsmath}                                  % Miscellaneous enhancements for mathematics
\usepackage{amssymb}                                  % Math symbols
\usepackage{booktabs}                                 % Extend tables
\usepackage{cite}                                     % Improved citation mechanics
\usepackage{graphicx}                                 % Enhance graphics
\usepackage{lineno}                                   % Add line numbers to page
\usepackage{microtype}                                % Uses different techniques for spacing
\usepackage{multicol}                                 % Create tables spanning multiple columns
\usepackage{multirow}                                 % Create tables spanning multiple rows
\usepackage{pgfplots}                                 % Plot in LaTeX
\usepackage{subcaption}                               % Subfigures (Get rid of this)
\usepackage{subfloat}                                 % Subfigures
\usepackage{tabularx}                                 % Add more contol to tables
\usepackage{tikz}                                     % Generate figures in LaTeX
\usepackage{xcolor}                                   % Colors
\usepackage{xfp}                                      % No trailing zeros
\usepackage{standalone}                               % Include standalone documents
\usepackage{hyperref}                                 % Better references (keep last)
%% Plot configurations
\usetikzlibrary{automata, positioning, arrows.meta}   % Tikz macros
\pgfplotsset{compat=1.3, width=\textwidth}
\graphicspath{ {img} }                                % Paths to find images
%% Paper configuration
\linenumbers                                          % Put line numbers
\setlength{\columnsep}{2cm}                           % Column separation
\let\ref\autoref                                      % Redifine `\ref` as `\autoref` because lazy
\let\cite\citep                                       % Redefine `\cite` as `\citep` because lazy
%%==============================================================================
% `autoref' formatting
\renewcommand*{\sectionautorefname}{Section}
\renewcommand*{\subsectionautorefname}{Subsection}
\renewcommand*{\subsubsectionautorefname}{Subsubsection}
\renewcommand*{\paragraphautorefname}{Paragraph}
%%===============================================================================
% Custom Commands
\newcommand{\TODO}[1]{{\color{green} To do: #1}}                                % For adding notes to paper
\def\keyFont{\fontsize{8}{11}\helveticabold }
%%-------------------------------------------------------------------------------
% Experiment parameters
% TODO: Ensure these are accurate
\newcommand{\A}{35 }                                                            % Number of buses
\newcommand{\N}{338 }                                                           % Number of visits
\newcommand{\acharge}{0.9}                                                      % BOD charge percentage
\newcommand{\bcharge}{0.7 }                                                     % EOD charge percentage
\newcommand{\mincharge}{25\% }                                                  % Min visit charge percent
\newcommand{\minchargeD}{0.25 }                                                 % Min visit charge decimal
\newcommand{\maxcharge}{100\% }                                                 % Max visit charge percent
\newcommand{\batsize}{388 }                                                     % Battery capacity
\newcommand{\fast}{15 }                                                         % Number of fast chargers
\newcommand{\slow}{15 }                                                         % Number of slow chargers
\newcommand{\fasts}{911 }                                                       % Speed of fast charger
\newcommand{\slows}{30 }                                                        % Speed of slow charger
\newcommand{\contvars}{7,511 }
\newcommand{\intvars}{328,282 }
%%-------------------------------------------------------------------------------
%% Solve output
\newcommand{\timeran}{1800 }                                                    % Time ran for MILP [s]
\newcommand{\gappercent}{65.2\% }                                               % Gap percent after runtime
\newcommand{\processor}{Ryzen 9 }                                               % Processor type
%%===============================================================================
% Authors
\def\keyFont{\fontsize{8}{11}\helveticabold }
\def\firstAuthorLast{Brown {et~al.}} %use et al only if is more than 1 author
\def\Authors{Alexander Brown\,$^{1,*}$, Greg Droge\,$^{1}$, Jacob Gunther\,$^{1}$}
% Affiliations should be keyed to the author's name with superscript numbers and be listed as follows: Laboratory,
% Institute, Department, Organization, City, State abbreviation (USA, Canada, Australia), and Country (without detailed
% address information such as city zip codes or street names). If one of the authors has a change of address, list the
% new address below the correspondence details using a superscript symbol and use the same symbol to indicate the author
% in the author list.
\def\Address{$^{1}$Department of Electrical and Computer Engineering, Logan, UT, USA}
% The Corresponding Author should be marked with an asterisk Provide the exact contact address (this time including
% street name and city zip code) and email of the corresponding author
\def\corrAuthor{Alexander Brown}
\def\corrEmail{A01704744@usu.edu}
\author[\firstAuthorLast ]{\Authors} %This field will be automatically populated
\address{} %This field will be automatically populated
\correspondance{} %This field will be automatically populated
\extraAuth{}
\date{}
\title{}
\hypersetup{
 pdfauthor={Alexander Brown},
 pdftitle={},
 pdfkeywords={},
 pdfsubject={},
 pdfcreator={Emacs 29.1 (Org mode 9.6.7)}, 
 pdflang={English}}
\begin{document}

\title{A Position Allocation Approach to the Scheduling of Battery Electric Bus Charging}

\onecolumn
\firstpage{1}

\maketitle

\begin{abstract}
  Robust charging schedules for an increasing interest of battery electric bus (BEB) fleets is a critical component to
  successful adoption. In this paper, a BEB charging scheduling framework that considers spatiotemporal schedule
  constraints, route schedules, fast and slow charging, and battery dynamics is modeled as a mixed integer linear
  program (MILP). The MILP is modeled after the berth allocation problem (BAP) in a modified form known as the position
  allocation problem (PAP). Linear battery dynamics are included to model the charging and discharging of buses while at
  the station and during their routes, respectively. The optimization coordinates BEB charging to ensure each BEB has
  sufficient charge while using slow chargers where possible for sake of battery health. The model also minimizes the
  total number of chargers utilized and prioritizes slow chargers. The model validity is demonstrated with a set of
  routes sampled from \TODO{Utah Transit Authority (UTA)} for \A buses and \N visits to the charging station. The model
  is also compared to a threshold based algorithm, referred to as the Quin-Modified method. The results presented show
  that the slow chargers are more readily selected and the charging and spatiotemporal constraints are met while
  considering the battery dynamics, minimizing the charger count, and consumption cost.

  \tiny \keyFont{ \section{Keywords:} Berth Allocation Problem (BAP), Position Allocation Problem (PAP), Mixed Integer
    Linear Program (MILP), Battery Electric Bus (BEB), Scheduling}
  % All article types: you may provide up to 8 keywords; at least 5 are mandatory.
\end{abstract}
\section{Introduction}
\label{sec:introduction}
The public transportation system is crucial in any urban area; however, the increased awareness and concern of
environmental impacts of petroleum based public transportation has driven an effort to reduce the pollutant footprint
\cite{de-2014-simul-elect,xylia-2018-role-charg,guida-2017-zeeus-repor-europ,li-2016-batter-elect}. Particularly,
the electrification of public bus transportation via battery power, i.e., battery electric buses (BEBs), has received
significant attention \cite{li-2016-batter-elect}. Although the technology provides benefits beyond reduction in
emissions, such as lower driving costs, lower maintenance costs, and reduced vehicle noise, battery powered systems
introduce new challenges such as larger upfront costs, and potentially several hours long ``refueling'' periods
\cite{xylia-2018-role-charg,li-2016-batter-elect}. Furthermore, the problem is exacerbated by the constraints of the
transit schedule to which the fleet must adhere, the limited amount of chargers available, and the adverse affects in
the health of the battery due to fast charging \cite{lutsey-2019-updat-elect}. This paper presents a continuous
scheduling framework for a BEB fleet that shares limited fast and slow chargers. This framework takes into consideration
linear charging dynamics and a fixed bus schedule while meeting a certain battery charge threshold throughout the day.

Many recent efforts have been made to simultaneously solve the problems of route scheduling, and charging fleets and
determining the infrastructure upon which they rely, e.g., \cite{wei-2018-optim-spatio,sebastiani-2016-evaluat-elect,hoke-2014-accoun-lithium,wang-2017-elect-vehic}. Several simplifications are made to make these problems
computationally feasible. These simplifications to the charge scheduling model include utilizing only fast chargers
while planning \cite{wei-2018-optim-spatio,sebastiani-2016-evaluat-elect,wang-2017-optim-rechar,zhou-2020-bi-objec,yang-2018-charg-sched,wang-2017-elect-vehic,qin-2016-numer-analy,liu-2020-batter-elect}. If slow chargers are used,
they are only employed at the depot and not the station \cite{he-2020-optim-charg,tang-2019-robus-sched}. Some
approaches also simplify by assuming a full charge is always achieved
\cite{wei-2018-optim-spatio,wang-2017-elect-vehic,zhou-2020-bi-objec,wang-2017-optim-rechar}. Others have assumed
that the charge received is proportional to the time spent on the charger
\cite{liu-2020-batter-elect,yang-2018-charg-sched}, which can be a valid assumption when the battery state-of-charge
(SOC) is below 80$\backslash$% charge \cite{liu-2020-batter-elect}.

This work builds upon the Position Allocation Problem \cite{qarebagh-2019-optim-sched}, a modification of the well
studied Berth Allocation Problem (BAP), as a means to schedule the charging of electric vehicles
\cite{buhrkal-2011-model-discr,frojan-2015-contin-berth,imai-2001-dynam-berth}. The BAP is a continuous time model
that solves the problem of allocating space for incoming vessels to be berthed. Each arriving vessel requires both time
and space to be serviced and thus must be carefully assigned a berthing location \cite{imai-2001-dynam-berth}. Vessels
are lined up parallel to the berth to be serviced and are horizontally queued as shown in \autoref{subfig:bapexample}.
The PAP utilizes this notion of queuing for scheduling vehicles to be charged, as shown in \autoref{subfig:papexample}.
The PAP is formulated as a rectangle packing problem by assuming that vehicle charging will take a fixed amount of time,
the amount of vehicles that can charge is limited by the physical width of the vehicles, and each vehicle visits the
charger a single time \cite{qarebagh-2019-optim-sched}.

The main contribution of this work is the extension of the PAP's novel approach to BEB charger scheduling. This includes
modeling and incorporation of a proportional charging model into the MILP framework, consideration of multiple charger
types, and inclusion of the route schedule for each bus. The last contribution is of importance because both the BAP and
PAP consider each arrival to be unique; thus, a method of tracking buses must be implemented. Input parameters are
selected in such a manner as to minimize the number of fast and slow charger utilized as well as minimize the power
consumption. The result is a MILP formulation that coordinates charging times and charger type for every visit that each
bus makes to the station while considering a dynamic charge model and scheduling constraints.

The remainder of the paper proceeds as follows: In \autoref{sec:the-position-allocation-problem}, the PAP is introduced
with a formulation of the resulting MILP. \autoref{sec:problemformulation} constructs the MILP for BEB scheduling,
including modifications to the PAP queuing constraints and development of a dynamic charging model.
\autoref{sec:example} demonstrates an example of using the formulation to coordinate \A buses over \N total visits to
the station. The paper ends in \autoref{sec:conclusion} with concluding remarks.
\section{The Position Allocation Problem}
\label{sec:the-position-allocation-problem}
This section provides a brief overview of the BAP and a detailed formulation of PAP as presented in
\cite{qarebagh-2019-optim-sched}.

\subsection{Overview of BAP}
\label{sec:overview-of-bap}
The BAP is a rectangle packing problem where a set of rectangles, \(\mathbb{O}\), are attempted to be optimally placed in
a larger rectangle, \(O\), as shown in \autoref{fig:packexample}. The rectangle packing problem is an NP-hard problem that
can be used to describe many real life problems \cite{bruin-2013-rectan-packin,murata-1995-rectan}. In some of these
problems, the dimensions of \(\mathbb{O}\) are held constant such as in the problem of packing modules on a chip, where
the widths and height of the rectangles represent the physical width and heights of the modules
\cite{murata-1995-rectan}. Other problems, such as the BAP can allow one side of the rectangle to vary depending on its
assigned position (e.g. the varying lengths of the vessels) \cite{buhrkal-2011-model-discr}.

The BAP solves the problem of optimally assigning incoming vessels to berth positions to be serviced
(\autoref{subfig:bapexample}). The width and height of \(O\) time horizon \(T\) and represent the berth length \(S\),
respectively. Similarly, the width and height for \(\mathbb{O}\) represent the time spent to service vessel \(i\) and the
space taken by docking vessel \(i\), respectively. In the BAP, the vessel characteristics (length of the vessel, arrival
time, handling time, desired departure time) are assumed to be known for all vessels to be serviced. A representation of
a BAP solution is shown in \autoref{fig:bap}. The x and y-axis represent time and space, respectively. The grey squares,
labeled A, B, C, and D, represent berthed vessels. The width of the boxes represent the time spent being serviced, and
the height represents the amount of space the vessel requires on the berth. The vertical line adjacent
 ``Arrival Time'' represents the actual
time that the vessel arrives and is available to be berthed. ``Berthing Time'' is the time the vessel is berthed and
begins being serviced. ``Completion time'' represents the time at which the berthing space becomes available again.

\subsection{The PAP Formulation}
\label{sec:the-pap-formulation}
The BAP forms the basis of the PAP; however, there are some differences in the way the variables are interpreted. For the
\(i^{th}\) visit, starting service time, \(u_i\), is now the starting charge time, the berth location, \(v_i\), is now the
charger queue for assignment, and the service time, \(s_i\), is now the elapsed charge time. There are also a few clarifying concepts
about how the system is modeled. The PAP models the set of chargers as one continuous line; that is, the natural
behavior of the PAP model is to allow vehicles to be queued anywhere along \([0,S]\). Similarly, the charge times are
continuous and can be placed anywhere on the time horizon, \([0,T]\), as long as the allocated times do not interfere with
other scheduled charge times. The PAP formulation's parameters can be divided into two categories: input parameters and
decision variables. Each type will now be introduced. The following parameters are constant inputs into the MILP.

\begin{itemize}
	\item $S$   : Size of the vehicle
	\item $T$   : time horizon
	\item $n_V$ : total number of incoming vehicles
	\item $s_i$ : charging time for vehicle $i;\; 1 \leq i \leq n_N$
	\item $s_i$ : width of vehicle $i;\; 1 \leq i \leq n_N$
	\item $a_i$ : arrival time of vehicle $i;\; 1 \leq i \leq n_N$
\end{itemize}

These constants define the problem bounds. The following list provides a series of decision variables used in the
formulation.

\begin{itemize}
    \item $u_i$    : starting charge time for vehicle $i;\; 1 \leq i \leq n_N$
    \item $v_i$    : assigned charge queue for vehicle $i;\; 1 \leq i \leq n_N$
    \item $d_i$    : departure time for vehicle $i;\; 1 \leq i \leq n_N$
    \item $\sigma_{ij}$ : binary variable that determines ordering of vehicles $i$ and $j$ in time
    \item $\psi_{ij}$ : binary variable that determines relative position of vehicles $i$ and $j$ when charging simultaneously
\end{itemize}

To determine the values for each of these decision variables, a MILP is formulated in \cite{qarebagh-2019-optim-sched}
and shown here for completeness.

\begin{equation}
	\label{eq:bapobjective}
	\min\; \sum_{i=1}^N (d_i - a_i)
\end{equation}

Subject to:
\begin{subequations}
\label{eq:bapconstrs}
\begin{align}
    u_j - u_i - s_i - (\sigma_{ij} - 1)T \geq 0                  \label{subeq:baptime}          \\
    v_j - v_i - s_i - (\psi_{ij} - 1)S \geq 0                \label{subeq:bapspace}           \\
    \sigma_{ij} + \sigma_{ji} + \psi_{ij} + \psi_{ji} \geq 1 \label{subeq:bapvalid_pos}     \\
    \sigma_{ij} + \sigma_{ji} \leq 1                              \label{subeq:bapsigma}        \\
    \psi_{ij} + \psi_{ji} \leq 1                              \label{subeq:bapdelta}        \\
    s_i + u_i = d_i                                               \label{subeq:bapdetach}       \\
    a_i \leq u_i \leq (T - s_i)                                   \label{subeq:bapvalid_starts} \\
    \sigma_{ij} \in \{0,1\},\;\psi_{ij} \in \{0,1\}\;           \label{subeq:bapsdspace}      \\
    v_i \in [0, S ]                                               \label{subeq:bapvspace}
\end{align}
\end{subequations}

\noindent

The objective function, \autoref{eq:bapobjective}, minimizes the time spent to service each vehicle by minimizing over
the sum of differences between the departure time, \(d_i\), and arrival time, \(a_i\). i.e., It seeks to get each vehicle
charged and on its way as quickly as possible.

\autoref{subeq:baptime}-\autoref{subeq:bapdelta} are used to ensure that individual rectangles do not overlap. For the
PAP, they ensure that two vehicles charging simultaneously are at different positions and, similarly, two vehicles that
have overlapping positions do not overlap temporally. \autoref{subeq:baptime} establishes temporal ordering when active
(\(\sigma_{ij}=1\)). Similarly, when \(\psi_{ij} =1\) in \autoref{subeq:bapspace} then spatial ordering is established. Constraints
\autoref{subeq:bapvalid_pos}-\autoref{subeq:bapdelta} enforce that spatial and/or temporal ordering is established
between each possible queue/vehicle pair. Constraints \autoref{subeq:bapsigma} and \autoref{subeq:bapdelta} enforce
consistency. For example, \autoref{subeq:bapsigma} enforces that vehicle \(i\) cannot come before vehicle \(j\) and vehicle
\(j\) simultaneously come before vehicle \(i\).

The last constraints force relationships between arrival time, charge start time, and departure time.
\autoref{subeq:bapdetach} states that the service start time, \(u_i\), plus the time to service vehicle \(i\), \(s_i\), must
equal the departure time, \(d_i\). \autoref{subeq:bapvalid_starts} enforces the arrival time, \(a_i\), to be less than or
equal to the service start time, \(u_i\), which in turn must be less than or equal to the latest time the vehicle may
begin to be serviced to stay within the time horizon. \autoref{subeq:bapsdspace} ensures that \(\sigma_{ij}\) and
\(\psi_{ij}\) are binary. \autoref{subeq:bapvspace} ensures that the assigned value of \(v_i\) is a valid charging position.
\section{A Rectangle Packing Formulation for BEB Charging}
\label{sec:problemformulation}
Applying the PAP to BEB charging requires four fundamental changes. The first is that the time that a BEB spends
charging is allowed to vary. Thus, \(s_i\) becomes a variable of optimization. Second, in the PAP each charging visit is
assumed to be a different vehicle. For the BEB charging problem, each bus may make multiple visits to the station
throughout the day and the resulting charge for a bus at a given time is dependent upon each of the prior visits made.
The third fundamental change is related to the first two. The charge of each bus must be tracked in
the optimization to ensure that charging across multiple visits is sufficient to allow each bus to execute its route throughout the day.
The final change in the PAP, the charger is one continuous bar with vehicle width effectively restricting the number of vehicles
charging simultaneously. For the BEB, it is assumed that a discrete number of chargers exist. Moreover, it is assumed
that these chargers have different charge rates.

The discussion of the four changes are separated into two sections. \autoref{sec:queuing} discusses the changes in the
spatial-temporal constraint formulation to form a queuing constraint. \autoref{sec:batt_dynamics} then discusses the
addition of the bus charge management. This section ends with a brief discussion of a modified objective function and
the statement of the full problem in \autoref{sec:BEB_MILP}. The notation is explained throughout and summarized in
\autoref{tab:variables}.

\subsection{Queuing Constraints}
\label{sec:queuing}
\noindent
The queuing constraints ensure that the busses entering the queues for charging are assigned in a feasible manner as
they come into the station. There are three sets to differentiate between different entities. \(\mathbb{B} = \{1, ...,
n_B\}\) is the set of bus indices with index \(b\) used to denote an individual bus, \(\mathbb{Q} = \{1, ..., n_Q\}\) is the set of
queues with index \(q\) used to denote an individual queue, and \(\mathbb{V} = \{1, ..., n_V\}\) is a set of visits to the
station with \(i,j\) used to refer to individual visits. The mapping \(\Gamma: \mathbb{V} \rightarrow \mathbb{B}\) is used to map a visit
index to a bus index with the shorthand \(\Gamma_i\) used to refer to the bus index for visit \(i\).

Most variables are now defined in terms of a visit. Two separate visits could correspond to different buses or visits by
the same bus. The PAP spatial variable, \(s_i\), is removed and \(v_i\) is made to be an integer corresponding to which
queue visit \(i\) will be using. Thus, when \(\psi_{ij} = 1\), vehicle \(i\) is queued to a charger that has a larger index than
the charger that vehicle \(j\) is queued, i.e., \(v_i-v_j \geq 1\). The variable \(S\) is likewise replaced with \(n_Q\). Note that
\(n_Q = n_B + n_C\), where \(n_B\) is the number of busses and \(n_C\) is the number of chargers. The rationale for having
more queues than chargers is to allow buses to sit idle instead of requiring the bus to charge at each visit. The
modified queuing constraints can be written as shown in \autoref{eq:packconstrs}.

\begin{subequations}
\label{eq:packconstrs}
\begin{align}
    v_i - v_j - (\psi_{ij} - 1)n_Q \geq 1       \label{subeq:space}        \\
    d_i \leq \tau_i                             \label{subeq:valid_depart} \\
    s_i \geq 0                               \label{subeq:pos_charge} \\
    v_i \in \mathbb{Q}                       \label{subeq:vspace}
\end{align}
\end{subequations}

The constraint in \autoref{subeq:space} is nearly identical to \autoref{subeq:bapspace}, but rather than viewing the
charger as a continuous strip of length \(S\), it is discretized into \(n_Q\) queues a width of unit length one. A BEB is
also assigned a unit length of one which is reflected in \autoref{subeq:space} by \(\cdot \geq 1\). \autoref{subeq:valid_depart}
ensures that the time the BEB is detatched from the charger, \(d_i\), is before its departure time, \(\tau_i\).
\autoref{subeq:vspace} defines the integer set of indices for queues for \(v_i\).

\subsection{Battery Charge Dynamic Constraints}
\label{sec:batt_dynamics}
Battery dynamic constraints are now introduced to relate busses to visits and guarantee that buses have sufficient time
to charge. Two constraints are enforced on the bus charge: busses must always have sufficient charge to execute their
respective routes and each bus must end the day with a specific charge threshold, preparatory for the next day.

The charge at the beginning of visit \(i\) is denoted as \(\eta_i\). As a charge on the bus is dependent upon the visits that
bus makes to the station, the mapping \(\Upsilon: \mathbb{V} \rightarrow \mathbb{V} \bigcup \{\varnothing\}\) is used to determine the next visit
that corresponds to the same bus, with \(\Upsilon_i\) being shorthand notation. Thus, \(\Gamma_i\) and \(\Gamma_{\Upsilon_i}\) would both map to the
same bus index as long as \(\Upsilon_i\) is not the null element, \(\varnothing\). That is, \(\Gamma_{\Upsilon_i}\) where \(\Upsilon_i = 0\) indicates
that there are no future visits for bus \(i\).

To drive time spent on the charger, \(s_i\), as well as define initial, final, and intermediate bus charges for each visit
\(i\), the sets for initial and final visits must be defined. Let the mapping of the first visit by each bus be denoted as
\(\Gamma^0_i : \mathbb{B} \rightarrow \mathbb{V}\). The indexed value of \(\Gamma^0_i\) represents the index for the first visit of bus \(b\) or
the null element, \(\varnothing\). Similarly, let \(\Gamma^f_i : \mathbb{B} \rightarrow \mathbb{B}\) contain the indexes for the final
visit of each bus \(b\) or the null element. The initial and final bus charge percentages, \(\alpha\) and \(\beta\), can then be
represented by the constraint equations \(\eta_{\Gamma^0_i} = \alpha \kappa_{\Gamma^0_i}\) and \(\eta_{\Gamma^f_i} = \beta \kappa_{\Gamma^f_i}\), respectively. The
intermediate charges must be determined at solve time.

It is assumed that the charge received is proportional to the time spent charging. The charge rate for charger \(q\) is
denoted as \(r_q\). Note that a value of \(r_q = 0\) corresponds to a queue where no charging occurs. A bus in such a queue
is simply waiting for the departure time. The queue indices are ordered such that the final \(n_B\) queues have \(r_q = 0\)
to allow an arbitrary number of buses to sit idle at any given moment in time. The amount of discharge between visits
\(i\) and \(\Upsilon_i\), the next visit of the same bus, is denoted as \(\Delta_i\). If visit \(i\) occurred at charger \(q\), the charge of
the bus coming into visit \(\Upsilon_i\) would be \(\eta_{\Upsilon_i} = \eta_i + s_i r_q - \Delta_i\).

The binary decision variable \(w_{iq}\) is introduced to determine whether visit \(i\) uses charger \(q\). This allows the
charge of the bus coming into visit \(\Upsilon_i\) to be written in summation form as

\begin{subequations}
    \label{subeq:pre_next_charge}
\begin{align}
    \eta_{\Upsilon_i} = \eta_i + \sum_{q=1}^{n_Q} s_i w_{iq} r_q - \Delta_i  \\
    \sum_{q=1}^{n_Q} w_{iq} = 1 \\
    w_{iq} \in \{0,1\}
\end{align}
\end{subequations}

The choice of queue for visit \(i\), becomes a slack variable and is defined in terms of \(w_{iq}\) as

\begin{equation}
    v_i = \sum_{q=1}^{n_Q} qw_{iq}
\end{equation}

Maximum and minimum values for the charges are included to ensure that the battery is not overcharged and to guarantee
sufficient charge for subsequent visits. The upper and lower battery charge bounds for bus \(b\) are \(\kappa_b\) and \(\nu_b \kappa_b\),
respectively0 \(\kappa_b\) is the battery capacity and \(\nu_b\) is a percent value. As \(\eta_i\) corresponds to the charge at the
beginning of the visit, the upper bound constraint must also include the charge received during the visit as follows.

\begin{subequations}
    \label{subeq:pre_min_max}
\begin{align}
    \eta_i + \sum_{q=1}^{n_Q} s_i w_{iq} r_q \leq \kappa_{\Gamma_i}                 \\
    \eta_i \geq \nu_{\Gamma_i} \kappa_{\Gamma_i}
\end{align}
\end{subequations}

Note that the term \(s_i w_{iq}\) is a bilinear term. A standard way of linearizing a bilinear term that contains an
integer variable is by introducing a slack variable with an either/or constraint
\cite[,rodriguez-2013-compar-asses]{chen-2010-applied}. Allowing the slack variable \(g_{iq}\) to be equal to \(s_i w_{iq}\),
\(g_{iq}\) can be defined as

\begin{equation}
    \label{eq:giq_cases}
    g_{iq} =
    \begin{cases}
        s_i & w_{iq} = 1 \\
        0 & w_{iq} = 0
    \end{cases}.
\end{equation}

\autoref{eq:giq_cases} can be expressed as a mixed integer constraint using big-M notation with the following four
constraints.

\begin{subequations}
    \label{eq:slack_gain}
\begin{align}
    s_i - (1 - w_{iq})M \leq g_{iq}  \label{subeq:repgpgret} \\
    s_i \geq g_{iq}                 \label{subeq:repgples} \\
    Mw_{iq} \geq g_{iq}              \label{subeq:repgwgret} \\
    0 \leq g_{iq}                   \label{subeq:repgwles}
\end{align}
\end{subequations}

\noindent
where \(M\) is a large value. If \(w_{iq} = 1\) then \autoref{subeq:repgpgret} and \autoref{subeq:repgples} become \(s_i \leq
g_{iq}\) and \(s_i \geq g_{iq}\), forcing \(s_i = g_{iq}\) with \autoref{subeq:repgwgret} being inactive. If \(w_{iq} = 0\),
\autoref{subeq:repgpgret} is inactive and \autoref{subeq:repgwgret} and \autoref{subeq:repgwles} force \(g_{iq} = 0\).

\subsection{The BEB Charging Problem}
\label{sec:BEB_MILP}
The goal of the MILP is to utilize chargers as little as possible to reduce energy costs with the fast charging being
penalized more to reduce battery damage. Thus, an assignment cost \(m_q\) and usage cost \(\epsilon_q\) are associated with each
charger, \(q\). These weights can be adjusted based on charger type or time of day that the visit occurs. The assignment
term takes the form \(w_{iq}m_q\), and the usage term takes the form \(g_{iq} \epsilon_q\). The resulting BEB charging problem is
defined in \autoref{eq:objective}.

\begin{equation}
\label{eq:objective}
	\min \sum_{i=1}^N \sum_{q=1}^{n_Q} \Big( w_{iq} m_q + g_{iq} \epsilon_q \Big) \\
\end{equation}

Subject to the constraints

\begin{multicols}{2}
\begin{subequations}
                                                     \label{eq:dynconstrs}
\begin{equation}
    u_i - u_j - s_j - (\sigma_{ij} - 1)T \geq 0              \label{subeq:m_time}         \\
\end{equation}
\begin{equation}
    v_i - v_j - (\psi_{ij} - 1)n_Q \geq 1                  \label{subeq:m_space}        \\
\end{equation}
\begin{equation}
    \sigma_{ij} + \sigma_{ji} + \psi_{ij} + \psi_{ji} \geq 1            \label{subeq:m_valid_pos}    \\
\end{equation}
\begin{equation}
    \sigma_{ij} + \sigma_{ji} \leq 1                              \label{subeq:m_sigma}        \\
\end{equation}
\begin{equation}
    \psi_{ij} + \psi_{ji} \leq 1                              \label{subeq:m_delta}        \\
\end{equation}
\begin{equation}
    s_i + u_i = d_i                                  \label{subeq:m_detach}       \\
\end{equation}
\begin{equation}
    \eta_{\Gamma^0_i} = \alpha \kappa_{\Gamma^0_i}                           \label{subeq:init_charge}    \\
\end{equation}
\begin{equation}
    a_i \leq u_i \leq (T - s_i)                            \label{subeq:m_valid_starts} \\
\end{equation}
\begin{equation}
    d_i \leq \tau_i                                        \label{subeq:m_valid_depart} \\
\end{equation}
\begin{equation}
    \eta_i + \sum_{q=1}^{n_Q} g_{iq} r_q - \Delta_i = \eta_{\gamma_i}   \label{subeq:next_charge}    \\
\end{equation}
\begin{equation}
    \eta_i + \sum_{q=1}^{n_Q} g_{iq} r_q - \Delta_i \geq \nu \kappa_{\Gamma_i} \label{subeq:min_charge}     \\
\end{equation}
\begin{equation}
    \eta_i + \sum_{q=1}^{n_Q} g_{iq} r_q \leq \kappa_{\Gamma_i}         \label{subeq:max_charge}     \\
\end{equation}
\begin{equation}
    \eta_{\Gamma^f_i} \geq \beta \kappa_{\Gamma^f_i}                          \label{subeq:final_charge}   \\
\end{equation}
\begin{equation}
    s_i - (1 - w_{iq})M \leq g_{iq}                     \label{subeq:gpgret}         \\
\end{equation}
\begin{equation}
    s_i \geq g_{iq}                                     \label{subeq:gples}          \\
\end{equation}
\begin{equation}
    Mw_{iq} \geq g_{iq}                                 \label{subeq:gwgret}         \\
\end{equation}
\begin{equation}
    0 \leq g_{iq}                                       \label{subeq:gwles}          \\
\end{equation}
\begin{equation}
    v_i = \sum_{q=1}^{n_Q} qw_{iq}                      \label{subeq:wmax}           \\
\end{equation}
\begin{equation}
    \sum_{q=1}^{n_Q} w_{iq} = 1                         \label{subeq:wone}           \\
\end{equation}
\begin{equation}
   w_{iq}, \sigma_{ij}, \psi_{ij} \in \{0,1\}\;            \label{subeq:binaryspace}        \\
\end{equation}
\begin{equation}
    v_i, q_i \in  \mathbb{Q}                                         \label{subeq:Qspace}        \\
\end{equation}
\begin{equation}
    i \in \mathbb{V}                                   \label{subeq:Ispace}         \\
\end{equation}
\end{subequations}
\end{multicols}

\autoref{subeq:m_time}-\autoref{subeq:m_valid_depart} are reiterations of the queuing constraints in
\autoref{eq:packconstrs}. \autoref{subeq:init_charge}-\autoref{subeq:final_charge} provide the battery charge
constraints. \autoref{subeq:gpgret} through \autoref{subeq:gwles} define the charge gain of every visit/queue
pairing. The last constraints \autoref{subeq:binaryspace}-\autoref{subeq:Ispace} define the sets of valid values for each
variable.
\section{Example}
\label{sec:example}
An example will now be presented to demonstrate the utility of the developed MILP charge scheduling techique. A
description of the scenario is first presented followed a description of an alternative heuristic based planning
strategy called Quin-Modified which is used as a comparison to the MILP PAP. Results are then presented for each of
planning strategies are presented, analyzed, and discussed.

\subsection{BEB Scenario}
\label{beb-scenario}
To display the capabilities of the model, an example scenario is presented. The scenario was run over a time horizon of
\(T=24\) hours, utilizes \(n_B = \A\) buses with \(n_V = \N\) visits to the station divided between the \(n_B\) buses. As stated
before, the route times are sampled from a set of routes from the UTA. Each bus has \batsize KWh battery that is
required to stay above \mincharge charge (\fpeval{\batsize * \minchargeD} KWh) to maintain battery health. Each bus is
assumed to begin the working day with \fpeval{\acharge*100}$\backslash$% charge (\fpeval{\acharge * \batsize} KWh). Additionally,
each bus is required to end the day with a minimum charge of \fpeval{\bcharge * 100}$\backslash$% (\fpeval{\bcharge * \batsize}
KWh). Each bus is assumed to discharge at a rate of 30 KW. Note that there are many factors that play a factor in the
rate of discharge; however, for the sake of simplicity an average rate is used. \(n_C = \fpeval{\fast + \slow}\) chargers
are utilized where \slow of the chargers are slow charging (\slows KW) and \fast are fast charging (\fasts KW). A
technique to minimize chargers will be employed.

To encourage the MILP PAP problem to utilize the fewest number of chargers, the value of \(m_q\) in the objective
function, \autoref{eq:objective}, is \(\forall q \in n_Q; m_q = 1000q\). The charge duration scalar, \(\epsilon_q\), is defined as
\(\epsilon_q = r_q\) to create a consumption cost term, \(g_{iq}\epsilon_q\). This is utilized to also encourage the model to minimize
active charger times, particuarly for the fast chargers.

Another heuristic-based optimization strategy, referred to as Quin-Modified, is also employed as a means of comparison
with the results of the MILP PAP. The Quin-Modified strategies is a based on the threshold strategy of
\cite{qin-2016-numer-analy}. The strategy has been modified slightly to accommodate the case of multiple charger types
and without exhaustive search for the best charger type. The heuristic is based on a set of rules that revolve around
the initial charge of the bus at visit \(i\). There are three different thresholds, low (60$\backslash$%), medium (75$\backslash$%), and high
(90$\backslash$%). Buses below the low threshold are prioritized to fast chargers then are allowed to utilize slow chargers if no
fast chargers are available. Buses between the low and medium threshold prioritize slow chargers first and utilize fast
chargers only if no slow chargers are available. Buses above the medium threshold and below high will only be assigned
to slow chargers. Buses above the high threshold will not be charged. Once a bus has been assigned to a charger, it
remains on the charger for the duration of the time it is at the station, or it reaches 90$\backslash$% charge, whichever comes
first.

UTA) bus routing data that occurs over a 24-hour time period. The total number of constraints resulted in \contvars
continuous and \intvars integer/binary constraints. The optimization was performed using the Gurobi MILP solver
\cite{gurobi-2021-gurob-optim} on a machine running an AMD Ryzen 9 5900X 12 - Processor (24 core) at 4.95GHz. The solver
was allowed to run for \timeran seconds and did not converge to the optimal result with a gap \gappercent.

\subsection{Results}
\label{results}
The schedule generated by the Quin-Modified strategy and the MILP PAP is shown in \autoref{subfig:quin-schedule} and
\autoref{subfig:milp-schedule}, respectively. The x-axis represents the time in hours. The y-axis represents the
assigned charging queue. Rows between 0 and \fpeval{\slow - 1} are active times for slow chargers, and rows in the range
of \fpeval{\slow - 1} and \fpeval{\fast + \slow - 1} are active times for fast chargers. The unique color/symbol styled
vertices represents the starting charge time for a bus \(b\) with the line to the vertical tick signifying the region of
time the charger is active. The line connecting points represent the charge sequence for a bus.

The first observation is in the choice of preferred chargers between the Quin-Modified and MILP scheduler. Looking at
\autoref{subfig:slow-charger-usage} and \autoref{subfig:fast-charger-usage}, the Quin-Modified schedule uses at most
four fast chargers and three slow at the same time, whereas the MILP schedule uses at most two fast chargers and four
slow at the same time. \autoref{subfig:milp-schedule} shows a few outliers for the selected fast chargers. This is most
likely due to the gap from the optimal solution. Both the Quin-Modifed and MILP schedule used the fast chargers in short
bursts (\textasciitilde{}0.2-0.5 hours). The main difference lies in the utilization strategy of the slow chargers. The Quin-Modified,
for the most part, opted for shorter bursts for the slow chargers (\textasciitilde{}0.3-0.7 hour), most heavily placed on the first slow
charger. The MILP also used these shorter charge times on the first slow charger; however, the schedule was able to
recognize the bus routes that had longer durations at the station and could choose the lower cost option, slow charging,
when available and of lower cost. Although one the MILP's objectives is to minimize the amount of chargers used, the
Quin-Modified schedule ended up using the same amount of total chargers as the MILP. The reason for this is the
construction of the objective function that is attempting to optimize over the consumption cost, total number of
chargers, and charger type. Hence, the objective function found it more efficient to utilize an extra slow charger for a
longer duration than to add another fast charger for a short duration. Furthermore, note the MILP schedule packed the
first queue for the fast and slow chargers more effectively than the Quin schedule. Although both schedules generated
are valid, no comparison of the quality the schedule can be made directly from \autoref{subfig:milp-schedule} and
\autoref{subfig:quin-schedule}.


\autoref{subfig:quin-charge} and \autoref{subfig:milp-charge} depicts the charge for every bus over the time horizon.
Every vehicle begins at 90$\backslash$% charge, finishes at \fpeval{\bcharge *100}$\backslash$% charge in the MILP PAP schedule, and never
goes below \mincharge in the intermediate arrivals as stated in the constraints \autoref{eq:dynconstrs}. There is no
guarantee for this in the Quin-Modified strategy which can be seen by some of the bus charges reaching a charge of 0$\backslash$%
as well as the distribution of final charges. The only sense of guarantee that the Quin-Modified supplies is its
predictability within the intermediate visits because of the heuristic nature (i.e. if the charge the low threshold, a
fast charger will be prioritized) whereas MILP places a bus in the queue that ``makes sense'' in respect to the larger
picture. The MILP PAP does not have an obvious sense of decision-making due to its weighted decisions that are affected
by the accumulation of decisions made prior.

Another important measure for the chargers is to compare the amount of power and energy consumed.
\autoref{fig:power-usage} depicts the power consumption throughout the time horizon. It can be seen that the
Quin-Modifed power consumption is steadily less or the same as the MILP schedule. This can be accounted for by the
MILP's constraints to keep the bus charges above \mincharge and to reach the \fpeval{\bcharge *100}$\backslash$% charge at the end
of the working day. It is also important to note that largest peak for the Quin-Modified schedule verses the lack of any
real peak for the MILP PAP schedule. Although the MILP PAP had firmer constraints than the Quin-Modified algorithm, it
maintained a steady power consumption profile throughout the time horizon. Along a similar vein, the accumulated energy
consumed is shown in \autoref{fig:energy-usage}. The MILP schedule is more efficient up until about hour 11. Again, this
can be accounted for by the fact the MILP is accommodating the extra constraints. Even with these constraints, MILP PAP
consumes about \(1\cdot10^5\) Kwh more than the Quin-Modified. The overlap of the MILP PAP can be accounted for by
\autoref{subfig:fast-charger-usage} and \autoref{subfig:slow-charger-usage}. Between the fifth and tenth hour, the MILP
schedule heavily uses slow chargers increasing the rate at which power is being consumed. Afterwards, the MILP schedule
at a minimum continues to use the same amount of chargers as the Quin Schedule. Again, due to the added constraints, the
MILP schedule must utilize more resources to keep within the specified bounds.
\section{Conclusion}
\label{sec:conclusion}
This work developed a MILP scheduling framework that optimally assigns slow and fast chargers to a BEB bus fleet
assuming a constant schedule. The BAP was introduced with an example formulation and was then compared to the PAP. The
PAP constructed on the BAP to allow the time spent on the charger, \(s_i\), to be a decision variable. Because the
original PAP required service time, \(s_i\), to be given, linear battery dynamics were introduced to drive charging times.
Additional constraints were also introduced to provide limits for the battery dynamics.

An example for the MILP PAP formulation was then presented and compared to a heuristic based schedule, referred to as
Quin-Modified. The MILP PAP optimization was run for \timeran seconds to a non-optimal solution with a gap of
\gappercent. The Quin-Modified and MILP schedule utilized similar amount of fast chargers; however, the MILP schedule
more heavily used the slow chargers to its advantage when the time was available to do so. More importantly, the MILP
PAP schedule utilized approximately \(1\cdot10^5\) Kwh more than the Quin-Modified, but the charges remained above the
constrained minimum charge of \mincharge, and charged all the buses to \fpeval{\bcharge *100}$\backslash$% at the end of the
working day. The Quin-Modified schedule, on the other hand, failed to keep all the bus charges above 0$\backslash$% throughout the
time horizon.

Further fields of interest are to utilize the formulation (\autoref{eq:objective} and \autoref{eq:dynconstrs}) with
nonlinear battery dynamics, calculation and utilization of the demand and consumption cost in the objective function,
and utilizing this formulation in a metaheuristic solver. Furthermore, ``fuzzifying'' the initial and final charge times
is of interest to allow flexibility in the arrival and departure times.

\bibliographystyle{Frontiers-Harvard}
\bibliography{/home/alex/NextCloud/Documents/docs/milp/milp-pap-paper-frontiers/citation-database/lit-ref,/home/alex/NextCloud/Documents/docs/milp/milp-pap-paper-frontiers/citation-database/lib-ref}

\nolinenumbers
\clearpage

\section{Figure Captions}
\label{sec:orgd1b8fc8}

\begin{subfigures}
    %%~~~~~~~~~~~~~~~~~~~~~~~~~~~~~~~~~~~~~~~~~~~~~~~~~~~~~~~~~~~~~~~~~~~~~~~~~~~~
    % BAP
    \begin{figure}[htpb]
    \centering
        \includestandalone{img/bap}
        \caption{Example of berth allocation. Vessels are docked in berth locations (horizontal) and are queued over
          time (vertical). The vertical arrow represents the movement direction of queued vessels and the horizontal
          arrow represents the direction of departure.}
        \label{subfig:bapexample}
    \end{figure}
    \hfill

    %%~~~~~~~~~~~~~~~~~~~~~~~~~~~~~~~~~~~~~~~~~~~~~~~~~~~~~~~~~~~~~~~~~~~~~~~~~~~~
    % PAP
    \begin{figure}[htpb]
    \centering
        \includestandalone{img/pap}
        \caption{Example of position allocation. Vehicles are placed in queues to be charged and move in the direction
          indicated by the arrow.}
        \label{subfig:papexample}
    \end{figure}
\end{subfigures}


\begin{table}[!htpb]
  \caption{Notation used throughout the paper}
  \label{tab:variables}
  \centering
  \begin{tabularx}{\textwidth}{l l}
    \toprule
    \textbf{Variable} & \textbf{Description}                                                                               \\
    \toprule
    \multicolumn{2}{l}{Input values}                                                                                       \\
    $n_B$        & Number of buses                                                                                         \\
    $M$          & An arbitrary very large upper bound value                                                               \\
    $n_V$        & Number of total visits                                                                                  \\
    $n_Q$        & Number of queues                                                                                        \\
    $n_C$ 	 & Number of chargers                                                                                      \\
    $\mathbb{V}$ & Set of visit indices, $\mathbb{V} = \{1, ..., n_V\}$                                                    \\
    $\mathbb{B}$ & Set of bus indices, $\mathbb{B} = \{1, ..., n_B\}$                                                      \\
    $\mathbb{Q}$          & Set of queue indices, $\mathbb{Q} = \{1, ..., n_Q\}$                                                             \\
    $i,j$        & Indices used to refer to visits                                                                         \\
    $b$ 	 & Index used to refer to a bus                                                                            \\
    $q$ 	 & Index used to refer to a queue                                                                          \\
    \hline
    \multicolumn{2}{l}{Problem definition parameters}                                                                      \\
    $\Gamma$   & $\Gamma: \mathbb{V} \rightarrow \mathbb{B}$ with $\Gamma_i$ used to denote the bus for visit $i$                                   \\
    $\alpha_i$ & Initial charge percentage time for visit $i$                                                                   \\
    $\beta_i$ & Final charge percentage for bus $i$ at the end of the time horizon                                             \\
    $\epsilon_q$ & Cost of using charger $q$ per unit time                                                                        \\
    $\Upsilon$   & $\Upsilon: \mathbb{V} \rightarrow \mathbb{V}$ mapping a visit to the next visit by the same bus with $\Upsilon_i$ being the shorthand. \\
    $\kappa_b$ & Battery capacity for bus $b$                                                                                   \\
    $\Delta_i$ & Discharge of visit over route $i$                                                                              \\
    $\nu_b$ & Minimum charge allowed for bus $b$                                                                             \\
    $\tau_i$ & Time visit $i$ must depart the station                                                                         \\
    $\zeta_b$ & Discharge rate for bus $b$                                                                                     \\
    $a_i$ & Arrival time of visit  $i$                                                                                     \\
    $i_0$ & Indices associated with the initial arrival for every bus in $A$                                               \\
    $i_f$ & Indices associated with the final arrival for every bus in $A$                                                 \\
    $m_q$ & Cost of a visit being assigned to charger $q$                                                                  \\
    $r_q$ & Charge rate of charger $q$ per unit time                                                                       \\
    \hline
    \multicolumn{2}{l}{Decision Variables}                                                                                 \\
    $\psi_{ij}$ & Binary variable determining spatial ordering of vehicles $i$ and $j$                                       \\
    $\eta_i$    & Initial charge for visit $i$                                                                                \\
    $\sigma{ij}$ & Binary variable determining temporal ordering of vehicles $i$ and $j$                                       \\
    $d_i$    & Ending charge time for visit $i$                                                                            \\
    $g_{iq}$ & The charge gain for visit $i$ from charger $q$                                                              \\
    $s_i$    & Amount of time spent on charger for visit $i$                                                               \\
    $u_i$    & Starting charge time of visit $i$                                                                           \\
    $v_i$    & Assigned queue for visit $i$                                                                                \\
    $w_{iq}$ & Binary assignment variable for visit $i$ to queue $q$                                                       \\
    \bottomrule
  \end{tabularx}
\end{table}

\begin{figure}[htpb]
\centering
    \includegraphics{img/spatiotemporal-packing}
    \caption{Example of rectangle packing problem.}
    \label{fig:packexample}
\end{figure}

\begin{figure}[ht]
\centering
    \includegraphics{img/baprep}
    \caption{The representation of the berth-time space}
    \label{fig:bap}
\end{figure}

\begin{figure}[htpb]
\centering
    \includegraphics{img/overlap}
    \caption{Examples of different methods of overlapping. Space overlap: $v_{k_1} < v_{i} + s_i \therefore \psi_{k_{1}i} = 0$.
             Time overlap $u_{k_1} < u_{j} + s_j \therefore \sigma_{k_{2}j} = 0$. Both space and time overlap $\sigma_{k_{3}i} = 0$ and
             $\psi_{k_{3}j} = 0$.}
    \label{fig:multipleassign}
\end{figure}

\begin{subfigures}
    %%~~~~~~~~~~~~~~~~~~~~~~~~~~~~~~~~~~~~~~~~~~~~~~~~~~~~~~~~~~~~~~~~~~~~~~~~~~~~
    % Quinn
    \begin{figure}[htpb]
    \centering
        \includegraphics{img/schedule-quinn}
        \caption{Charging schedule generated by Quin Modified algorithm.}
        \label{subfig:quin-schedule}
    \end{figure}

    \hfill

    %%~~~~~~~~~~~~~~~~~~~~~~~~~~~~~~~~~~~~~~~~~~~~~~~~~~~~~~~~~~~~~~~~~~~~~~~~~~~~
    % MILP
    \begin{figure}[htpb]
    \centering
        \includegraphics{img/schedule-milp}
        \caption{Charging schedule generated by MILP PAP algorithm.}
        \label{subfig:milp-schedule}
    \end{figure}
\end{subfigures}

\begin{subfigures}
    %%~~~~~~~~~~~~~~~~~~~~~~~~~~~~~~~~~~~~~~~~~~~~~~~~~~~~~~~~~~~~~~~~~~~~~~~~~~~~
    % Quinn
    \begin{figure}[htpb]
    \centering
        \includegraphics{img/charge-quinn}
        \caption{Bus charges for the Quin Modified charging schedule. The charging scheme of the Quin charger is more predictable during the working day.}
        \label{subfig:quin-charge}
    \end{figure}

    \hfill

    %%~~~~~~~~~~~~~~~~~~~~~~~~~~~~~~~~~~~~~~~~~~~~~~~~~~~~~~~~~~~~~~~~~~~~~~~~~~~~
    % MILP
    \begin{figure}[htpb]
    \centering
        \includegraphics{img/charge-milp}
        \caption{The bus charges for the MILP PAP charging schedule. The MILP model allows for guarantees of minimum/maximum changes during the working day as well as charges at the end of the day.}
        \label{subfig:milp-charge}
    \end{figure}
\end{subfigures}

\begin{figure}[htpb]
\centering
    \includegraphics{img/power}
    \caption{Total accumulated energy consumed by the Quin-Modified and MILP schedule throughout the time horizon.}
    \label{fig:power-usage}
\end{figure}

\begin{figure}[htpb]
\centering
    \includegraphics{img/energy}
    \caption{Amount of power consumed by Quin-Modified and MILP schedule over the time horizon.}
    \label{fig:energy-usage}
\end{figure}

\begin{subfigures}
    %%~~~~~~~~~~~~~~~~~~~~~~~~~~~~~~~~~~~~~~~~~~~~~~~~~~~~~~~~~~~~~~~~~~~~~~~~~~~~
    % Fast
    \begin{figure}[htpb]
    \centering
        \includegraphics{img/charger-count-fast}
        \caption{Number of fast chargers for Quin and MILP PAP.}
        \label{subfig:fast-charger-usage}
    \end{figure}

    \hfill

    %%~~~~~~~~~~~~~~~~~~~~~~~~~~~~~~~~~~~~~~~~~~~~~~~~~~~~~~~~~~~~~~~~~~~~~~~~~~~~
    % Slow
    \begin{figure}[!ht]
    \centering
        \includegraphics{img/charger-count-slow}
        \caption{Number of slow chargers for Quin and MILP PAP.}
        \label{subfig:slow-charger-usage}
    \end{figure}
\end{subfigures}
\end{document}
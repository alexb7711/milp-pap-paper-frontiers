% Created 2023-08-02 Wed 11:42
% Intended LaTeX compiler: pdflatex
\documentclass[utf8]{FrontiersinHarvard}
\usepackage[utf8]{inputenc}
\usepackage[T1]{fontenc}
\usepackage{graphicx}
\usepackage{longtable}
\usepackage{wrapfig}
\usepackage{rotating}
\usepackage[normalem]{ulem}
\usepackage{amsmath}
\usepackage{amssymb}
\usepackage{capt-of}
\usepackage{hyperref}
%%==============================================================================
% Package configuration
\usepackage{amsmath}                                  % Miscellaneous enhancements for mathematics
\usepackage{amssymb}                                  % Math symbols
\usepackage{booktabs}                                 % Extend tables
\usepackage{cite}                                     % Improved citation mechanics
\usepackage{graphicx}                                 % Enhance graphics
\usepackage{lineno}                                   % Add line numbers to page
\usepackage{microtype}                                % Uses different techniques for spacing
\usepackage{multicol}                                 % Create tables spanning multiple columns
\usepackage{multirow}                                 % Create tables spanning multiple rows
\usepackage{pgfplots}                                 % Plot in LaTeX
\usepackage{subcaption}                               % Subfigures (Get rid of this)
\usepackage{subfloat}                                 % Subfigures
\usepackage{tabularx}                                 % Add more contol to tables
\usepackage{tikz}                                     % Generate figures in LaTeX
\usepackage{xcolor}                                   % Colors
\usepackage{xfp}                                      % No trailing zeros
\usepackage{standalone}                               % Include standalone documents
\usepackage{hyperref}                                 % Better references (keep last)
%% Plot configurations
\usetikzlibrary{automata, positioning, arrows.meta}   % Tikz macros
\pgfplotsset{compat=1.3, width=\textwidth}
\graphicspath{ {img} }                                % Paths to find images
%% Paper configuration
\linenumbers                                          % Put line numbers
\setlength{\columnsep}{2cm}                           % Column separation
\let\ref\autoref                                      % Redifine `\ref` as `\autoref` because lazy
\let\cite\citep                                       % Redefine `\cite` as `\citep` because lazy
%%==============================================================================
% `autoref' formatting
\renewcommand*{\sectionautorefname}{Section}
\renewcommand*{\subsectionautorefname}{Subsection}
\renewcommand*{\subsubsectionautorefname}{Subsubsection}
\renewcommand*{\paragraphautorefname}{Paragraph}
%%===============================================================================
% Custom Commands
\newcommand{\TODO}[1]{{\color{green} To do: #1}}                                % For adding notes to paper
\def\keyFont{\fontsize{8}{11}\helveticabold }
%%-------------------------------------------------------------------------------
% Experiment parameters
% TODO: Ensure these are accurate
\newcommand{\A}{35 }                                                            % Number of buses
\newcommand{\N}{338 }                                                           % Number of visits
\newcommand{\acharge}{0.9}                                                      % BOD charge percentage
\newcommand{\bcharge}{0.7 }                                                     % EOD charge percentage
\newcommand{\mincharge}{25\% }                                                  % Min visit charge percent
\newcommand{\minchargeD}{0.25 }                                                 % Min visit charge decimal
\newcommand{\maxcharge}{100\% }                                                 % Max visit charge percent
\newcommand{\batsize}{388 }                                                     % Battery capacity
\newcommand{\fast}{15 }                                                         % Number of fast chargers
\newcommand{\slow}{15 }                                                         % Number of slow chargers
\newcommand{\fasts}{911 }                                                       % Speed of fast charger
\newcommand{\slows}{30 }                                                        % Speed of slow charger
\newcommand{\contvars}{7,511 }
\newcommand{\intvars}{328,282 }
%%-------------------------------------------------------------------------------
%% Solve output
\newcommand{\timeran}{1800 }                                                    % Time ran for MILP [s]
\newcommand{\gappercent}{65.2\% }                                               % Gap percent after runtime
\newcommand{\processor}{Ryzen 9 }                                               % Processor type
%%===============================================================================
% Authors
\def\keyFont{\fontsize{8}{11}\helveticabold }
\def\firstAuthorLast{Brown {et~al.}} %use et al only if is more than 1 author
\def\Authors{Alexander Brown\,$^{1,*}$, Greg Droge\,$^{1}$, Jacob Gunther\,$^{1}$}
% Affiliations should be keyed to the author's name with superscript numbers and be listed as follows: Laboratory,
% Institute, Department, Organization, City, State abbreviation (USA, Canada, Australia), and Country (without detailed
% address information such as city zip codes or street names). If one of the authors has a change of address, list the
% new address below the correspondence details using a superscript symbol and use the same symbol to indicate the author
% in the author list.
\def\Address{$^{1}$Department of Electrical and Computer Engineering, Logan, UT, USA}
% The Corresponding Author should be marked with an asterisk Provide the exact contact address (this time including
% street name and city zip code) and email of the corresponding author
\def\corrAuthor{Alexander Brown}
\def\corrEmail{A01704744@usu.edu}
\author[\firstAuthorLast ]{\Authors} %This field will be automatically populated
\address{} %This field will be automatically populated
\correspondance{} %This field will be automatically populated
\extraAuth{}
\date{}
\title{A Position Allocation Approach to the Scheduling of Battery Electric Bus Charging}
\hypersetup{
 pdfauthor={Alexander Brown},
 pdftitle={A Position Allocation Approach to the Scheduling of Battery Electric Bus Charging},
 pdfkeywords={},
 pdfsubject={},
 pdfcreator={Emacs 28.2 (Org mode 9.6.7)}, 
 pdflang={English}}
\begin{document}

\maketitle
\onecolumn
\firstpage{1}

\begin{abstract}
  Robust charging schedules for an increasing interest of battery electric bus (BEB) fleets is a critical component to
  successful adoption. In this paper, a BEB charging scheduling framework that considers spatiotemporal schedule
  constraints, route schedules, fast and slow charging, and battery dynamics is modeled as a mixed integer linear
  program (MILP). The MILP is modeled after the berth allocation problem (BAP) in a modified form known as the position
  allocation problem (PAP). Linear battery dynamics are included to model the charging and discharging of buses while at
  the station and during their routes, respectively. The optimization coordinates BEB charging to ensure each BEB has
  sufficient charge while using slow chargers where possible for sake of battery health. The model also minimizes the
  total number of chargers utilized and prioritizes slow chargers. The model validity is demonstrated with a set of
  routes sampled from \TODO{Utah Transit Authority (UTA)} for \A buses and \N visits to the charging station. The model
  is also compared to a threshold based algorithm, referred to as the Quin-Modified method. The results presented show
  that the slow chargers are more readily selected and the charging and spatiotemporal constraints are met while
  considering the battery dynamics, minimizing the charger count, and consumption cost.

  \tiny \keyFont{ \section{Keywords:} Berth Allocation Problem (BAP), Position Allocation Problem (PAP), Mixed Integer
    Linear Program (MILP), Battery Electric Bus (BEB), Scheduling} %All article types: you may provide up to 8 keywords;
  at least 5 are mandatory.
\end{abstract}

\section{Introduction}
\label{sec:introduction}
The public transportation system is crucial in any urban area; however, the increased awareness and concern of
environmental impacts of petroleum based public transportation has driven an effort to reduce the pollutant footprint
\cite{de-2014-simul-elect, xylia-2018-role-charg, guida-2017-zeeus-repor-europ, li-2016-batter-elect}. Particularly, the
electrification of public bus transportation via battery power, i.e., battery electric buses (BEBs), has received
significant attention \cite{li-2016-batter-elect}. Although the technology provides benefits beyond reduction in
emissions, such as lower driving costs, lower maintenance costs, and reduced vehicle noise, battery powered systems
introduce new challenges such as larger upfront costs, and potentially several hours long ``refueling'' periods
\cite{xylia-2018-role-charg, li-2016-batter-elect}. Furthermore, the problem is exacerbated by the constraints of the
transit schedule to which the fleet must adhere, the limited amount of chargers available, and the adverse affects in
the health of the battery due to fast charging \cite{lutsey-2019-updat-elect}. This paper presents a continuous
scheduling framework for a BEB fleet that shares limited fast and slow chargers. This framework takes into consideration
linear charging dynamics and a fixed bus schedule while meeting a certain battery charge threshold throughout the day.

Many recent efforts have been made to simultaneously solve the problems of route scheduling, and charging fleets and
determining the infrastructure upon which they rely, e.g., \cite{wei-2018-optim-spatio, sebastiani-2016-evaluat-elect,
  hoke-2014-accoun-lithium, wang-2017-elect-vehic}. Several simplifications are made to make these problems
computationally feasible. These simplifications to the charge scheduling model include utilizing only fast chargers
while planning \cite{wei-2018-optim-spatio, sebastiani-2016-evaluat-elect, wang-2017-optim-rechar, zhou-2020-bi-objec,
  yang-2018-charg-sched, wang-2017-elect-vehic, qin-2016-numer-analy,liu-2020-batter-elect}. If slow chargers are used,
they are only employed at the depot and not the station \cite{he-2020-optim-charg,tang-2019-robus-sched}. Some
approaches also simplify by assuming a full charge is always achieved
\cite{wei-2018-optim-spatio,wang-2017-elect-vehic,zhou-2020-bi-objec,wang-2017-optim-rechar}. Others have assumed that
the charge received is proportional to the time spent on the charger \cite{liu-2020-batter-elect,yang-2018-charg-sched},
which can be a valid assumption when the battery state-of-charge (SOC) is below 80\% charge
\cite{liu-2020-batter-elect}.

This work builds upon the Position Allocation Problem \cite{qarebagh-2019-optim-sched}, a modification of the well
studied Berth Allocation Problem (BAP), as a means to schedule the charging of electric vehicles
\cite{buhrkal-2011-model-discr,frojan-2015-contin-berth,imai-2001-dynam-berth}. The BAP is a continuous time model that
solves the problem of allocating space for incoming vessels to be berthed. Each arriving vessel requires both time and
space to be serviced and thus must be carefully assigned a berthing location \cite{imai-2001-dynam-berth}. Vessels are
lined up parallel to the berth to be serviced and are horizontally queued as shown in \autoref{subfig:bapexample}. The
PAP utilizes this notion of queuing for scheduling vehicles to be charged, as shown in \autoref{subfig:papexample}. The
PAP is formulated as a rectangle packing problem by assuming that vehicle charging will take a fixed amount of time, the
amount of vehicles that can charge is limited by the physical width of the vehicles, and each vehicle visits the charger
a single time \cite{qarebagh-2019-optim-sched}.

The main contribution of this work is the extension of the PAP's novel approach to BEB charger scheduling. This includes
modeling and incorporation of a proportional charging model into the MILP framework, consideration of multiple charger
types, and inclusion of the route schedule for each bus. The last contribution is of importance because both the BAP and
PAP consider each arrival to be unique; thus, a method of tracking buses must be implemented. Input parameters are
selected in such a manner as to minimize the number of fast and slow charger utilized as well as minimize the power
consumption. The result is a MILP formulation that coordinates charging times and charger type for every visit that each
bus makes to the station while considering a dynamic charge model and scheduling constraints.

The remainder of the paper proceeds as follows: In \autoref{sec:positionallocationproblem}, the PAP is introduced with a
formulation of the resulting MILP. \autoref{sec:problemformulation} constructs the MILP for BEB scheduling, including
modifications to the PAP queuing constraints and development of a dynamic charging model. \autoref{sec:example}
demonstrates an example of using the formulation to coordinate \A buses over \N total visits to the station. The paper
ends in \autoref{sec:conclusion} with concluding remarks.

\bibliographystyle{Frontiers-Harvard}
\bibliography{citation-database/lit-ref.bib, citation-database/lib-ref.bib}

\nolinenumbers
\clearpage
\end{document}